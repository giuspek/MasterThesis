\chapter*{Summary} % senza numerazione
\label{sommario}

Propositional Satisfiability (SAT) is the problem of determining if there exists an interpretation that satisfies a given Boolean formula. An extension to the previous problem is represented by Satisfiability Modulo Theories (SMT), which focuses on deciding the satisfiability of first-order formulas with respect to various theories, such as real and integer arithmetic, bit vector and floating-point arithmetic. In the last decade, multiple SAT and SMT solvers have been developed to address these problems and have been applied in real-life applications, mainly in the formal verification field. \\
Given their computation complexity, both problems have been extensively studies to determine efficient approaches in retrieving the satisfiability of formulas. One of the most recent and promising suggestions involves the exploit of Adiabatic Quantum Computing: the Boolean problem is converted into an Ising problem, which can be represented by Quantum architectures. The annealer then uses this encoding to search the final state with minimum stage, with the goal of obtaining  a valid truth assignment. The mentioned reduction, known as SAT-to-Ising, is still in its primordial stages. Starting from the studies , we will concentrate on determining the weaknesses and limitations of the current implementation, suggesting novel approaches and fixes to enhance and reduce issues caused by quantum phenomena, such as co-tunneling.

\newpage

\addcontentsline{toc}{chapter}{Sommario} % da aggiungere comunque all'indice





