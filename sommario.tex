\chapter*{Summary} % senza numerazione
\label{sommario}

Satisfiability Modulo Theories (SMT) is the problem of deciding the satisfiability of first-order formulas with respect to various theories, such as real and integer arithmetic, bit vector and floating-point arithmetic. In the last decade, multiple SMT solvers have been developed to address these problems and have been applied in real-life applications, mainly in the formal verification field. \\
Some problem instances require not only to obtain a valid assignment of variables satisfying the formulas, but they also need to retrieve a model which is optimum with respect to a subset of objective functions. For instance, when timed and hybrid systems are taken into consideration, studying the minimum interval of time before an event happens can be useful to ensure safety properties are always valid. In order to deal with these problems, an extension of SMT was proposed known as Optimization Modulo Theories (OMT), adding to the SMT formulation the already cited objective functions to optimize. \\
While SMT has been extensively studied and its state-of-the-art is quite advanced, OMT is still in its first steps; only a few OMT solvers are available (one of them, OptiMathSAT, is developed in collaboration by UniTN and FBK) and a lot of progress has still to be done. In this thesis we will concentrate on the problem of effectively encoding SAT and MaxSAT problems into Quantum Annealers. The completion of this task is achieved using OMT solvers to determine the mapping of a Boolean Formula into qubits of the annealer, thus showing a real-life application of OMT. In particular we will discuss the problem of quantum co-tunneling, how it negatively affects the encoding provided by the solvers and a proposed approach to contrast this issue. \\

\newpage

\addcontentsline{toc}{chapter}{Sommario} % da aggiungere comunque all'indice





