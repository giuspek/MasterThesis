\thispagestyle{empty}

\begin{center}
  {\bf \Huge Ringraziamenti}
\end{center}

\vspace{3cm}


\emph{
  In questi 5 anni di vita trentina il mio motto è stato quello di portare caos nel mondo e alle persone che mi circondano, il che ha reso la vita decisamente poco noiosa. Mentre scrivevo questi ringraziamenti ho pensato di portare questo spirito: da una parte vi rendo la lettura più elettrizzante, dall'altra mi semplifico la vita giocandomi un po' di sarcasmo. \\
  Prima di tutto ringrazio tutti i miei familiari e, in particolare, i miei genitori: niente frase fatta "per avermi supportato e sopportato", semmai grazie per avermi ascoltato al telefono mentre mi preoccupavo di esami e concorsi vari, senza mai avermi chiuso la telefonata in faccia. Può sembrare poco, ma visto che mi preoccupo sempre 7/24, in realtà ha la sua rilevanza. \\
  Un ringraziamento speciale va poi a mio fratello Ciro, che nelle ultime settimane di stesura non solo mi ha nutrito di "meme" e varie, ma è stato anche forzato ad imparare un po' di LaTex per sistemare il layout e permettermi di guadagnare tempo nella revisione finale. Grazie per aver imparato cosa significa soffrire con l'allineamento delle immagini, ti sei anticipato un po' di stress per la tua futura tesi. \\
  Ringrazio poi tutti gli amici che ho conosciuto in questi anni e che mi hanno vissuto in prima persona. Comincio ringraziando quelle che son state le mie dirimpettaie del primo anno, Stefania e Giulia: tra voli d'angelo, Cup Song e speciali preghiere pre-presentazione di progetti siete state le prime con cui ho condiviso la vita a Trento e abbiamo creato tantissimi ricordi uno più divertente dell'altro. Grazie per essere sopravvissute alla mia indecenza. \\
  Ringrazio i miei coinquilini storici Emanuale ed Emiliano (a.k.a. Hulele e MercatiniDiNatale): prima di tutto grazie per avermi sopportato quando passavo i tre quarti della giornata steso sul divano o quando dimenticavo di strizzare la maledetta spugnetta. Oltre la mera pazienza però abbiamo condiviso tante serate e conversazioni in quel salone, tra una partita a Fifa o Mario Kart. Non potevo chiedere coinquilini migliori in questi anni. \\
  Ovviamente devo ringraziare i miei due compagni di progetti in questi anni. Da una parte ringrazio Luca per essere stato il mio front-end developer di fiducia e aver reso belli i miei algoritmi di backend. Sei riuscito a farmi appassionare alla parte più elettronica (come quel timer con Arduino); inoltre sei sempre stato un ottimo amico a cui raccontare le mie vicende di vita. Certo, pensi troppo a "fatturare", ma ti si vuole bene proprio perchè sei così. Dall'altra ringrazio "il Debo" (perchè chiamarti Andrea mi fa troppo strano), con il quale ho condiviso gli ultimi progetti ed esami. Le lacrime versate dietro le richieste di Yannis, o i giorni con gli esercizi impossibili di Lo Cigno fanno di noi dei reduci di questa università e sono contento che la nostra fatica sia stata ricompensata. Dovessi scegliere, continuerei a far progetti con voi. \\
  Devo ringraziare tutti i "Povo rangers", il gruppo di compagni di corso con i quale ho speso gli ultimi anni. Siamo stati nerd quanto basta e i nostri gruppi di studio ci hanno lasciato tante perle. Non dimenticherò mai il baccano mattutino nel Povobus e i versi contro i poveri studenti costretti a prendere il 5/. \\
  Infine, ma non per questo meno importante, ringrazio il mio eterno compagno di avventure Zeno: nonostante spesso ti dica che sei l'opposto della responsabilità, mi hai aiutato davvero tanto a capire meglio gli altri e a tirare fuori il meglio di me. Grazie per avermi insegnato che spesso bisogna riflettere meno e "buttarsi" di più nelle cose: senza di te i Top 20, l'Erasmus in Finlandia e la slitta con gli Husky in Lapponia, nonchè tutti i futuri progetti lavorativi in grande non esisterebbero. 
}
