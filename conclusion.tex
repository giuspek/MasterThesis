\chapter{Conclusions}
\label{cha:conclusioni}

From the analysis discussed in the second part of this thesis, we see how the development of a postprocessing procedure can benefit the quantum encoding and provides more stable encoding. The most interesting success is represented by the capability of obtaining new penalty functions w simply re-arranging some qubits

\section{Future work}

\subsection{Enhancing CP-to-OMT}

A second future direction to take into account would focus on bridging the gap between Constraint Programming and Optimization Modulo Theories. The open-source interface described in the previous chapters can be extended to efficiently manage each MiniZinc structure and procedure. To cover the full extent of CP standards some additions are required, such as defining an encoding based on Floating-Point Numbers or dealing with non-linear constraints and objectives (in this case the first future direction will help in completing the task). This may require a general review of the current implementation of the FlatZinc interface, so as to become modular and easy to extend.
Moreover, we can improve the situation by developing a T-solver for the theory of sets, which is currently managed using a quite inefficient mix of Boolean and arithmetic constraints. 
To conclude, another interesting issue concerns CP global constraints, constraints that capture a relation between a non-fixed number of variables. In particular, it would be necessary to study approaches to integrate dedicated procedures to the SMT and OMT framework emulating these constraints. \\
 



