\chapter{Optimization Modulo Theories and Constraint Programming}
\label{cha:introOMT}

\section{Bridging CP and OMT}

OMT presents some specific affinities that make it a valid candidate to deal with CSP: the availability of decision procedures for infinite-precision arithmetic, the efficient combination of
Boolean and arithmetical reasoning and the ability to produce conflict explanations are shared between the two paradigms and it is not a case both are used to deal with formal verification problems. Achieving this task benefits the progression of the current state-of-the-art: the comparison between OMT solvers and CP tools on problems that do not belong to their original application domain extends their application in novel fields and speed up the process of finding critical points for future research. \\ 
Despite that, transforming a CP instance into a OMT problem is not trivial; in particular there are multiple valid formulation of a CP problem into an OMT instance. Using one option instead of the others requires particular attention, since this choice could drastically affect performances (as shown in []). This situation represents the first issue hardening the bridging task. In addition to this difficulty, there are some MiniZinc features that requires a careful mapping to the SMT-LIB standard. In particular:

\begin{itemize}
    \item FlatZinc support three basic scalar types (int, float and bool) and two compound types (set and array). Compound types do not have a direct correspondence in the SMT-LIB standard; moreover integer and float can be represented using finite precision (using respectively bit-vector and floating point arithmetic) or applying the linear arithmetic theory.
    \item MiniZinc introduced local and global constraints to express complex relations among variables. OMT solver currently do not introduce ad hoc decision procedures to manage them efficiently.
    \item MiniZinc is capable of dealing with non-linear and transcendental functions, such as the logarithm or the cosine function. On the other hand, not every solver built on SMT-LIB support them.
\end{itemize}

\pagebreak
